\section{Lineare Zeitinvariante System (LTI) \buchSeite{20}}
\subsection{Laplace-Transformation \buchSeite{20}}


 	\subsubsection{Eigenschaften \buchSeite{22}}
  		\renewcommand{\arraystretch}{2}
  				\begin{tabular}{|l|l|}
  		        	\hline
  		        	Linearität & 
  		 			$\alpha\cdot f(t) + \beta\cdot g(t) \; \laplace \; \alpha\cdot F(s) + \beta\cdot
  		 			G(s)$ \\
  		 			\hline
  		 			"Ahnlichkeit / Streckung &
  		 			$f(\alpha t) \; \laplace \; \frac{1}{\alpha}F \left (\frac{s}{\alpha} \right ) \quad 0
  		 			<\alpha \in\mathbb{R}$ \\
  		 			\hline
  		 			Faltung im Zeitbereich &
  		 			$f(t) \ast g(t) = \int\limits_{0}^{t} f(\tau)g(t-\tau)d\tau \; \laplace \; F(s)
  		 			\cdot G(s)$\\
  		 			\hline
  		 			Faltung im Frequenzbereich &
  		 			$f(t) \cdot g(t) \; \laplace \; \frac{1}{2\pi j}\int\limits_{c-j\infty}^{c+j\infty}
  		 			F(\xi) G(s-\xi)d\xi$ \\
  		 			\hline
  		 			Ableitung im Zeitbereich &
  		 			$\frac{\partial f(t)}{\partial t} \; \laplace \; sF(s)
  		 			-f(0^+)$ \\
  		 			\hline
  		 			Ableitungen im Zeitbereich &
  		 			$\frac{\partial^n f(t)}{\partial t^n} \; \laplace \; s^nF(s)
  		 			-s^{n-1}f(0^+)-s^{n-2}\frac{\partial f(0^+)}{\partial t}-\ldots
  		 			-s^0\frac{\partial^{n-1} f(0^+)}{\partial t^{n-1}}$ \\
  		 			\hline
  		 			Multiplikation mit $t$ &
  		 			$t\cdot f(t)  \; \laplace \; \frac{-\partial F(s)}{\partial s}$ \\
  		 			\hline
  		 			Ableitung im Frequenzbereich &
  		 			$(-t)^n f(t) \; \laplace \;  \frac{\partial^n F(s)}{\partial s^n}$ \\
  		 			\hline
  		 			Verschiebung im Zeitbereich nach rechts &
  		 			$f(t - t_0) \; \laplace \; F(s)e^{- t_0 s}$ \\
  		 			\hline
  					Verschiebung im Zeitbereich nach links &
  					$f(t + t_0) \; \laplace \; e^{t_0 s} \cdot [F(s) - \int\limits_0^{t_0} f(t) \cdot e^{-st} dt]$\\
  					\hline
  		 			Verschiebung im Freq.bereich (Dämpfungssatz) &
  		 			$f(t)e^{\mp\alpha t} \; \laplace \; F(s\pm\alpha)$ \\
  		 			\hline
  		 			Integration (Sprungantwort)&
  		 			$\int\limits_0^t f(\tau)d\tau \; \laplace \; \frac{F(s)}{s}$ \\
  		 			\hline
  		 			Anfangswert &
  		 			$\lim\limits_{t\rightarrow 0^+} f(t) = \lim\limits_{s\rightarrow \infty} s\cdot F(s),\text{~wenn
  		 			}  \lim\limits_{t\rightarrow 0} f(t)\text{~existiert}.$ \\
  		 			\hline
  		 			Endwert &
  		 			$\lim\limits_{t\rightarrow \infty} f(t) = \lim\limits_{s\rightarrow 0} s\cdot F(s),\text{~wenn
  		 			}  \lim\limits_{t\rightarrow \infty} f(t)\text{~existiert}.$ \\
  		 			\hline
  		       	\end{tabular}\\
  		  \renewcommand{\arraystretch}{1.8}
  				
	\subsubsection{Sprungantwort} 
	Integration der Impulsantwort $G_1(s) = \frac{3+s}{s}$\\ 
	\[\Rightarrow \quad \frac{1}{s}\cdot G_1(s)= \frac{1}{s}\cdot\frac{3+s}{s}=\frac{3}{s^2}+\frac{1}{s} \quad \laplace \quad 3t +\sigma(s)\]
	
	\subsection{Transformation einer periodischen Funktion \buchSeite{24}}
		\[ F(s) = F_0(s)\cdot\dfrac{1}{1 - e^{-sT}} \qquad \text{mit } F_0(s) \text{ als eine Periode}\] 


\newpage
	\subsubsection{Laplace-Tabelle \buchSeite{21}}
	\begin{multicols}{2}
		\begin{center}
			\begin{tabular}{|lcc|}
				\hline
				$\delta \left( t \right)$ & $\laplace$ & $1$ \\
				$\delta \left( t - a \right)$ & $\laplace$ & $e^{- a s}$\\
				$\sigma \left( t \right)$ = $\varepsilon \left( t \right)$ & $\laplace$ & $\frac{1}{s}$ \\
				$\sigma \left( t \right) \cdot t$ & $\laplace$ & $\frac{1}{s^2}$\\
				$\sigma \left( t \right) \cdot t^2$ & $\laplace$ & $\frac{2}{s^3}$\\
				$\sigma \left( t \right) \cdot t^n$ & $\laplace$ & $\frac{n!}{s^{n+1}}$\\
				$\sigma \left( t \right) \cdot e^{a t}$ & $\laplace$ &
				$\frac{1}{s-a}$\\
				$\sigma \left( t \right) \cdot t \cdot e^{a t}$ & $\laplace$ &
				$\frac{1}{( s - a )^2}$\\
				$\sigma \left( t \right)\cdot t^2 \cdot e^{a t}$ &
				$\laplace$ & $\frac{2}{{( s - a )}^3}$\\
				$\sigma \left( t \right)\cdot t^n \cdot e^{ a t}$ &
				$\laplace$ & $\frac{n!}{(s-a)^{n+1}}$\\
				$\sigma \left( t \right) \cdot (1-e^{a t}$) & $\laplace$ &
				$\frac{- a}{s ( s - a )}$\\
				$\sigma \left( t \right) \cdot \cos \left(b t \right)$ & $\laplace$ &
				$\frac{s}{ s^2 + b^2}$\\
				$\sigma \left( t \right) \cdot \sin \left(b t \right)$ & $\laplace$ &
				$\frac{b}{s^2 + {b^2}}$\\
				$\sigma \left( t \right) \cdot e^{a t} \cdot \cos \left(b t \right)$ & $\laplace$ &
				$\frac{s- a}{ (s- a)^2 + b^2}$\\
				$\sigma \left( t \right) \cdot e^{a t} \cdot \sin \left(b t \right)$ & $\laplace$ &
				$\frac{b}{(s- a)^2 + {b^2}}$\\
				$\sigma \left( t \right) \cdot \frac{e^{a t} - e^{b t}}{a-b}$ & $\laplace$ & $\frac{1}{(s-a)(s-b)}$ \\
				$\sigma \left( t \right) \cdot \frac{a e^{a t} - b e^{b t}}{a-b}$ & $\laplace$ & $\frac{s}{(s-a)(s-b)}$\\
				\hline
			\end{tabular}
		\end{center}
	\columnbreak
		\begin{center}
	\[\boxed{F(s)=\int\limits_0^\infty f(t)e^{-st}dt} \qquad s=\sigma+j\omega\]\\
\subsubsection{Bedingungen}
\begin{itemize}
		\item Definitionsbereich nur für kausale Systeme $t\geq 0$\\
		\item Integrierbar über das Intervall $(0,\infty)$\\
		\item Wachstum kleiner als der von einer Exponentialfunktion\\ 
		\item Gegen"uber $j\omega$ bei der Fourier-Transformation ist bei der
			Laplace-Transformation $s$ verallgemeinert zu $s=\sigma + j\omega$. Das
			bedeutet, dass die Fourier-Transformierte $F(j\omega)$ durch die
			Laplace-Transformation $F(s)$ ausgedr\"uckt werden kann. \\
		\item mit $\sigma = 0 \rightarrow$ Amplitude bleibt konstant\\
		\item mit $\sigma > 0 \rightarrow$ explodiert die Amplitude f\"ur $0 < t \rightarrow \infty$ \\
		\item mit $\sigma < 0 \rightarrow$ klingt die Amplidute für $0 < t \rightarrow \infty$ auf $0$ ab
\end{itemize}

	\subsubsection{Vorgehen Rücktransformation \buchSeite{21}}
		\begin{enumerate}
			\item Kürzen oder vereinfachen
			\item Rücktransformation mittels Laplace-Tabelle
			\item Partialbruchzerlegung falls nötig
			\item $h(t)\hspace{0.2cm}\underline{nicht} < 0$
		\end{enumerate}
	\end{center}
\end{multicols}

\subsection{Differentialgleichungen lösen\buchSeite{25-28}}
Die ‘interne’ Beschreibung transformieren, fallen Probleme mit Anfangsbedingungen weg, da man näher
an den physikalischen Grössen bleibt
\begin{multicols}{2}
		\begin{center}
			\includegraphics[width=4cm]{./images/diffgleichung.png}
		\end{center}
		$Y(s) = \frac{s^2 LC+sRC}{s^2 LC +sRC +1}\cdot U(s) + \frac{[sLC + RC] \cdot u_{C}(0) + L \cdot i(0)}{s^2 LC +sRC +1}$
	\columnbreak
\begin{center}
	\begin{tabular}{ll}
		$i(t) = C \cdot \dot{u}_{C}(t)$ & $I(s) = C  \cdot  [s  \cdot  U_{C}(s) - u_{C}(0)]$  \\
		$u_{L}(t) = L \cdot \dot{i}(t)$ & $U_{L}(s) = L  \cdot  [s  \cdot  I(s) - i(0)]$ \\
		$u_{R}(t) = R \cdot i(t)$ & $U_{R}(s) = R  \cdot  I(s)$ \\
		$u(t) = u_{C}(t) + u_{R}(t) + u_{L}(t)$ & $U(s) = U_{C}(s) + U_{R}(s) + U_{L}(s)$ \\
		$y(t) = u_{R}(t) + u_{L}(t)$ & $Y (s) = U_{R}(s) + U_{L}(s)$ \\
	\end{tabular}
\end{center}
\end{multicols}


%Gleichungssystem mit 5 Gleichungen und 6 Signalen. Das Eliminieren der 4 internen Signale $(U_{C}, U_{R}, U_{L}, I)$ fällt leichter, da weniger Anfangsbedingungen nötig sind, als wenn nur 1 Gleichung im Zeitbereich in den Bildbereich transformiert wird.\\
%Beispiel DGL: $u-4\dot{u}=3e+4\dot{e}+\ddot{e} \ \laplace \ U(1-4s)=E(3+4s+s^2) \qquad \frac{Ausgang}{Eingang}=\frac{U}{E}=\frac{3+4s+s^2}{1-4s}$

\subsection{Übertragungsfunktion UTF \buchSeite{29}}
\hspace{2.3cm}G(s)\\
$Y(s) = \underbrace{\frac{\overbrace{s^2 LC+sRC}}{s^2 LC +sRC +1}\cdot U(s)} + \underbrace{\frac{[sLC + RC] \cdot u_{C}(0) + L \cdot i(0)}{s^2 LC +sRC +1}}$ \newline
\textcolor{white}{x} \hspace{2.4cm} $Y_{E}(s)$ \hspace{3.8cm} $Y_{F}(s)$ \\
Das Ausgangssignal Y (s) setzt sich aus den beiden Termen $Y_{E}(s)$ und $Y_{F}(s)$ zusammen.Der erzwungene Anteil $Y_{E}(s)$ ist durch das Eingangssignal U(s) bestimmt, aber unabhängig von den Anfangsbedingungen des Netzwerks.
Beim freien Anteil $Y_{F}(s)$ ist es umgekehrt; dieser hängt nur von den Anfangsbedingungen,
nicht aber vom Eingangssignal ab. Die freie Antwort zeigt also, wie
das System sich verhält, wenn man es ‘sich selbst überlässt’
für $u_{C}(0) \neq 0$ und/oder $i_{L}(0) \neq 0$.
%\begin{itemize}
%	\item $ y_{F}(t) die Form y_{F,1} \cdot e^{\lambda_{1}t} + y_{F,2} \cdot e^{\lambda_{1}t} \text{ haben muss, wobei}$
%	\item $ \lambda_{1,2}= \frac{-RC \pm \sqrt{(RC)^2 - 4 \cdot LC} }{2 \cdot LC}$ die Wurzeln des charakteristischen Polynoms sind,
%	\item und die Werte $y_{F,1} \text{ und } y_{F,2}$ sich aus den Anfangsbedingungen ergeben.
%\end{itemize}
Offensichtlich ist die freie Antwort $Y_{F}$ dann von Bedeutung, wenn man an ganz konkreten
Signalwerten interessiert ist, sonst wird der Fokus jedoch auf $Y_{E}$ gelegt. \\
%Die erzwungene Antwort $Y_{E}(s) = G(s) \cdot U(s)$ kann als Produkt geschrieben werden, wobei G(s) der Übertragungs-funktion entspricht.

\subsection{Darstellung und Eigenschaften von UTF \buchSeite{31}}

\begin{eqnarray}
G(s)=\frac{Z(s)}{N(s)}& = &\frac{b_{m}s^m + b_{m-1}s^{m-1}+ ... + b_{1}s+b_{0}}{s^{n}+a_{n-1}s^{n-1} + ... + a_{2}s^{2}+a_{1}s + a_{0}} \\ 
& = & K\cdot \frac{(s-z_{1})(s-z_{2})...(s-z_{m})}{(s-p_{1})(s-p_{2})...(s-p_{n})} \qquad \text{mit} \quad K=b_{m} \\ 
& = & \frac{c_{1}}{s-p_{1}} + \frac{c_{2}}{s-p_{2}} + ...  + \frac{c_{n}}{s-p_{n}}
\end{eqnarray}

\subsubsection{minimalphasig \buchSeite{43}}
wenn \textbf{keine Nullstelle $z_i$ in der rechten Halbebene liegt} und die \textbf{Totzeit $T_t = 0$} ist.
Motivation für den Ausdruck ‘minimalphasig’: bei stabilen Systemen hat ein
minimalphasiges System bei gegebenem Amplitudengang den minimal möglichen
Phasenabfall im Phasengang.\\
Merke: Nichtminimalphasige Systeme holen in falsche Seite aus. Wert der Sprungantwort wechselt kurz das Vorzeichen.

\subsubsection{Reaslisierbarkeit \buchSeite{33}}
%UTF mit $m > n$ weist differentierendes Verhalten weisen, solche mit m = n proportionales Verhalten; der ‘Rest’ mit $m < n$ ist
%ein Tiefpass. 
Systeme mit \textbf{m $\leq$ n} werden als realisierbar bezeichnet. Motiviert wird
diese Bezeichnung dadurch, weil für ein Schrittsignal am Eingang ein P-Glied ideal
funktionieren kann, ein D-Glied aber nicht. Praktisch bedeutet das, dass bei einem Regler nicht mehr Nullstellen als Pole
eingebaut werden dürfen. $\Rightarrow \quad m > n$ ist nicht realisierbar

\subsubsection{Minimalität \buchSeite{33}}

Eine UTF mit Nennergrad n ist minimal, wenn \textbf{keine Pol-Nullstellenkürzung
möglich} ist. Um die UTF zu realisieren, sind dann mindestens n Integratoren nötig. Zu beachten ist, dass Kürzungen in
der rechten Halbebene normalerweise nicht erlaubt sind, da durch sie Instabilitäten
in einem System unsichtbar werden.

\subsubsection{Stabilität \buchSeite{34}}
\begin{itemize}
\item  ist asymptotisch stabil, wenn
alle Pole $p_i$ in der linken Halbebene liegen (ohne Imaginärachse).
\item  ist grenzstabil, wenn
\begin{itemize}
\item kein Pol in der rechten Halbebene liegt und
\item mindestens ein Pol auf der Imaginärachse liegt und
\item alle Pole auf der Imaginärachse einfache Pole sind.
\end{itemize}
\item  ist instabil, wenn
sie weder asymptotisch stabil noch grenzstabil ist.
%\item  ist stabil, wenn sie asymptotisch stabil ist.
\end{itemize}

\subsection{Partialbruchzerlegung (PBZ) \buchSeite{31}}
%\subsubsection{Allgemeines Vorgehen}			\[f(x)=\frac{x^2+20x+149}{x^3+4x^2-11x-30} \Rightarrow \; \Rightarrow x^{3}+4x^{2}-11x-30=(x+2)(x^{2}+2x-15)=(8x+2)(x+5)(x-3)\]
%			Ansatz:
%			\[f(x)=\frac{x^2+20x+149}{x^3+4x^2-11x-30}=\frac{A}{x-3} + \frac{B}{x+2} + \frac{C}{x+5}=
%			\frac{A(x+2)(x+5)+B(x-3)(x+5)+C(x-3)(x+2)}{(x-3)(x+2)(x+5)}\]
%			Gleichungssystem \textbf{(Z"ahler gleichsetzen)} aufstellen mit beliebigen $x_i$-Werten (am Besten Polstellen oder 0,1,-1 w"ahlen):
%			\[\begin{array}{l}x_1=3:\;-9+60+149=A\cdot5\cdot8\;\;\;\Rightarrow A=5\\
%			x_2=-2:\;-4-40+149=B(-5)\cdot3\; \Rightarrow B=-7\\
%			x_3=-5:\;-25-100+149=C(-8)(-3) \Rightarrow C=1 \end{array} \Rightarrow f(x)=\frac{5}{x-3}+\frac{7}{x+2} + \frac{1}{x+5}\]
%			weitere Ans"atze f"ur andere Typen von Termen: (Mehrere Werte f"ur $x$ verwenden, auch wenn kein Koeffizient 0 wird.)
%			\[f(x)=\frac{5x^2-37x+54}{x^3-6x^2+9x}=\frac{A}{x}+\frac{B}{x-3}+\frac{C}{(x-3)^2}=\frac{A(x-3)^2+Bx(x-3)+Cx}{x(x-3)^2}\]
%			\[f(x)=\frac{1,5x}{x^3-6x^2+12x-8}=\frac{A}{x-2}+\frac{B}{(x-2)^2}+\frac{C}{(x-2)^3}=\frac{A(x-2)^2+B(x-2)+C}{(x-2)^3}\]
%			\[f(x)=\frac{x^2-1}{x^3+2x^2-2x-12}=\frac{A}{x-2}+\frac{Bx+C}{x^2+4x+6}=\frac{A(x^2+4x+6)+(Bx+C)(x-2)}{(x-2)(x^2+4x+6)}\]
\subsubsection{Bedingungen}
\begin{itemize}
\item Bedingung 1: In der gezeigten Form kann die PBZ nur durchgeführt werden,
wenn die Pole paarweise verschieden sind: $p_{i} \neq p_{j}$ falls i $\neq$ j.
\item Bedingung 2: In der gezeigten Form kann die PBZ nur durchgeführt werden, wenn m < n. Die Schrittantwort des Systems ist dann stetig.
\item Bei komplex-konjugierten Polen $p_{j} = p^{\ast}_{i}$ ergeben sich komplex-konjugierte Faktoren
$c_{j} = c^{\ast}_{i}$ . Werden die entsprechenden zwei Terme addiert, ergeben sich
für die Summe reelle Koeffizienten (PT2-Glied):
\end{itemize}
\subsubsection{Standardfall}

\[ G(s) =\frac{c_{1}}{s+3} + \frac{c_{2}}{s+5} + \frac{c_{3}}{s+7} \quad \text{ mit } \quad {c_{i}= \frac{Z(s)}{\frac{dN(s)}{ds}}\mathop{\Bigg|_{s=p_{i}}}}\]

Beispiel: \[\frac{22s +35}{(s+1)(s+2)}=\frac{c_{1}}{s+1}+\frac{c_{2}}{s+2}=\frac{\frac{22s+35}{2s+3}\mathop{\Big|_{s=-1}}}{s+1}+\frac{\frac{22s+35}{2s+3}\mathop{\Big|_{s=-2}}}{s+2}=\frac{13}{s+1}+\frac{9}{s+2}\]
kann durch PBZ in drei parallel geschaltete PT1-Glieder umgeformt werden. Die
Faktorisierung ist dabei nur beim Nenner nötig.

\subsubsection{Nichtminimale Übertragungsfunktion}
\[G(s) =\frac{8s^{2} + 80s + 200}{s^3 + 15s^2 + 71s + 105}= \frac{4}{s+3} + \frac{0}{s+5} + \frac{4}{s+7}\]
In der faktorisierten Form äussert sich dies durch eine Pol-/Nullstellenkürzung; in der PBZ ist der entsprechende Koeffizient $c_{2}$ = 0. Die UTF ist nicht minimal; eine vollständig gekürzte UTF dagegen heisst minimal. \\

\subsubsection{Übertragungsfunktion mit komplexen Polen}
\[G(s) =\frac{8s^{2} + 70s + 134}{s^3 + 15s^2 + 81s + 175}= \frac{8s^2+70s+134}{(s+7)(s+4+3j)(s+4-3j)}=\frac{2}{s+7}+\underbrace{\frac{3+2j}{s+4-3j}+\frac{3-2j}{s+4+3j}}\]
\textcolor{white}{x} \hspace{14.5cm} $\frac{6s+12}{s'2+8s+25}$\\

Grundsätzlich kann G(s) wieder als Parallelschaltung dreier PT1-Glieder aufgefasst
werden. Glieder mit komplexen Werte für Verstärkung und Zeitkonstante werden
üblicherweise aber zu Teilsystemen zweiter Ordnung zusammengefasst.

\subsubsection{Fall m $>$ n}
%UTF mit m $>$ n haben bei hohen Frequenzen differentierendes Verhalten, mit m = n
%haben sie bei hohen Frequenzen Proportionalverhalten. 
Um solche UTF in eine
parallele Form zu bringen, kann mit einer Polynomdivision gearbeitet werden. Die Polynomdivision wird abgebrochen, bevor negative s-Potenzen entstehen. 
%Terme mit nicht-negativen s-Potenzen haben proportionales
%bzw. differentierendes Übertragungsverhalten; bei höheren s-Potenzen träte auch
%mehrfache Differentiation auf. 
Wenn ein echt gebrochenes Polynom übrig bleibt, so kann dieses mit PBZ weiter zerlegt werden.



\subsection{Stabilitätskriterien, Routh-Kriterium \buchSeite{38}}

Eine notwendige Stabilitätsbedingung für das charakteristische Polynom besteht darin, dass alle Koeffizienten positiv sind. (Vorzeichenbedingung) $\Rightarrow$ Bei 1. und 2. Ordnung \textbf{hinreichend!}
\begin{equation}
\boxed{N(s) = a_{n}s^n + a_{n-1}s^{n-1} + . . . + a_2s^2 + a_1s + a_0 \quad mit \quad a_n > 0}
\end{equation}
Sind bei einem Polynom alle Koeffizienten negativ (inklusive $a_n$), ist die obige
notwendige Bedingung sinngemäss ebenfalls erfüllt.



Da dies keine hinreichende Bedingung ist muss bei bestehen des Kriteriums mit positiven Koeffizienten auch das Routh Kriterium erfüllt sein.
\begin{itemize}
	\item Die Tabelle hat n + 1 Zeilen mit den Indizes n . . . 0.
	\item In die beiden oberen Zeilen werden die Koeffizienten ai gefüllt. Je nach Systemordnung
	kommt dabei der letzte Koeffizient a0 in die erste oder zweite Zeile.
	
	\item {Jede der weiteren Zeilen der Tabelle ergibt sich durch algebraisches Rechnen
	mit den Einträgen der zwei darüber liegenden Zeilen. Das Muster der Berechnung
	wiederholt sich dabei in jeder Zeile.\\
	\begin{tabularx}{\textwidth}{XXX}
	$b_1=\frac{a_{n-1}a_{n-2}-a_{n}a_{n-3}}{a_{n-1}}$
	& $b_2=\frac{a_{n-1}a_{n-4}-a_{n}a_{n-5}}{a_{n-1}}$
	& $b_3=\ldots$ \\
	$c_1=\frac{b_{1}a_{n-3}-a_{n-1}b_{2}}{b_{1}}$
	& $c_2=\frac{b_{1}a_{n-5}-a_{n-1}b_{3}}{b_{1}}$
	& $c_3=\ldots$ \\
	$d_1=\frac{c_{1}b_{2}-b_{1}c_{2}}{c_{1}}$
	& $d_2=\ldots$ & \\
	\end{tabularx}}
\end{itemize}

\begin{minipage}{9cm}
	\begin{itemize}
		\item Durch dieses Vorgehen ergibt sich eine Dreiecksstruktur; die Tabelle endet mit
		genau einem Eintrag in Zeile 0.
		\item Die Auswertung der Tabelle besteht durch Inspektion der \textbf{(grau unterlegten)}
		ersten Kolonne: dass alle Koeffizienten dieser \textbf{Kolonne positiv} sind, ist eine
		notwendige und hinreichende Bedingung für (asymptotische) Stabilität.
	\end{itemize}
\end{minipage}
\hspace{0.5cm}
\begin{minipage}{6cm}
	\begin{tabular}{|l||l|l|l|l|l|l|}
	\hline
		n & \cellcolor{hellgrau}$a_n$ & $a_{n-2}$ & $a_{n-4}$ & $\ddots\vdots$ & $a_{2}$  & $a_{0}$ \\ \hline
		n-1 & \cellcolor{hellgrau} $a_{n-1}$ & $a_{n-3}$ & $a_{n-5}$ & $\vdots\ddots$ & $a_{1}$  & 0 \\ \hline\hline
		n-2 & \cellcolor{hellgrau} $b_{1}$ & $b_{2}$ & $b_{3}$ & $\cdots$ & $b_{n/2}$  & 0 \\ \hline
		n-3 & \cellcolor{hellgrau} $c_{1}$ & $c_{2}$ & $\cdots$ & $c_{(n-2)/2}$ & 0 & 0 \\ \hline
		n-4 & \cellcolor{hellgrau} $d_{1}$ & $\ddots$ & $\ddots$ & $\ddots$ & 0 & 0 \\ \hline
		$\vdots$ & \cellcolor{hellgrau} $\vdots$ &  $\ddots$ & $\ddots$ & $\ddots$ & 0 & 0 \\ \hline
		0 & \cellcolor{hellgrau} $x_1$ & 0 & 0 & 0 & 0 & 0 \\ \hline
	\end{tabular}
\end{minipage}



\subsubsection{Dominante Pole und Nullstellen \buchSeite{38}}

\begin{equation}
\boxed{Y(s)=\frac{K}{s(sT+1)}=K\mathop{\Big[}\frac{1}{s}-\frac{1}{s+\frac{1}{T}}\mathop{\Big]} \laplace K\mathop{\Big[}1-e^{\frac{-1}{T}}\mathop{\Big]}=y(t)}
\end{equation}

	\begin{figure}[h!]	
		\begin{center}
			\includegraphics[width=10cm]{./images/PoleNullstellen.png}
		\end{center}
		\caption{Schrittantwort eines PT1-Glieds mit UTF $\frac{K}{sT + 1}$ (links) und Pollage bei $p_1 = -1/T$ (rechts)}
	\end{figure}
um eine schnelle Reaktion der Strecke
zu erreichen, sind grosse Werte am Reglerausgang nötig. Das bedeutet, dass man
T nicht beliebig klein machen kann, bzw. den Pol $p_1$ nicht beliebig weit nach links
legen darf.\\

Das $PT_2$-Glied hat eine UTF mit den drei Parametern K, $\xi$ und T (bzw. $\omega_n$ = 1/T ).
\begin{equation}
\boxed{\frac{Y(s)}{R(s)}=G(s)=\frac{K}{s^2T^2+2\xi Ts+1}=\frac{K\omega^{2}_{n}}{s^2 + 2\xi \omega_{n}s+\omega^{2}_{n}}}
\end{equation}
%\begin{itemize}
%\item Modifikation
%von K verändert die Höhe der Schrittantwort durch Strecken/Stauchen in
%vertikaler Richtung. (Signalwerte)
%\item Modifikation von T (bzw. $\omega n$) verändert
%die Schrittantwort durch Strecken/Stauchen in horizontaler Richtung. (Zeit)
%\item Abhängig von $\xi$ ergeben sich folgende Fälle:
%\begin{itemize}
%\item Fall mit $\xi$ < 0
%Da T als positiv angenommen wird, ergibt sich für $\xi$ < 0 gemäss Vorzeichenbedingung
%ein instabiles System.
%\item Fall mit $\xi$ = 0
%Dies ergibt einen harmonischen Oszillator; die Pole liegen bei $\pm j\omega_{n}$.
%NB: die Periode der Oszillation entspricht nicht der Zeitkonstanten T.
%\item Fall mit $ 0 < \xi < 1$
%Hier resultiert ebenfalls eine Schwingung mit einer Periode; allerdings ist
%die Schwingung gedämpft. Dieser Fall wird unten genauer betrachtet.
%\item Fall mit $1 \leq \xi$
%Dies entspricht dem aperiodischen Fall. Für (35) ist eine Faktorisierung
%mit reellen Werten T1,2 möglich:
%Für $\xi$ = 1 ist T1 = T2 = T; das $PT_2$-Glied entspricht der Serieschaltung
%zweier identischer PT1-Glieder.
%Für $\xi$ > 1 ist $T_1 > T > T_2$; das $PT_2$-Glied entspricht der Serieschaltung
%zweier unterschiedlicher PT1-Glieder.
%Für $\xi \gg 0$ wird mit $T_1 \gg T_2$ die Zeitkonstante T1 dominant. Das $PT_2$-
%Glied verhält sich ähnlich wie ein $PT_1$-Glied
%\end{itemize}
%\end{itemize}
%
%\[p_{1,2}=\frac{-2\xi\omega_n \pm \sqrt{4\xi^2 \omega^2_n - 4\omega^2_n}}{2}=\underbrace{-\xi\omega_n} \pm \underbrace{\sqrt{\xi^2-1}\cdot \omega_n}\]
%\textcolor{white}{x} \hspace{10.5cm} $\sigma$ 
%\hspace{1.5cm} $j\omega$
%\begin{eqnarray}
%\sigma=-\xi\omega_n \quad und \quad \omega=\sqrt{1-\xi^2}\cdot \omega_n  \quad bzw. \\ \omega_n=\sqrt{\omega^2 + \sigma^2}  \quad und  \quad \xi = -\frac{\sigma}{\omega_n}=-\frac{\sigma}{\sqrt{\omega^2 + \sigma^2}}\label{xiOmega}
%\end{eqnarray}
%
%Die Schrittantwort eines $PT_2$-Glieds mit $|\xi| < 1$ ergibt
%\begin{equation}
%\boxed{Y(s)=\frac{K\omega^2_n}{s(s^2+2\xi\omega_n s +\omega^2_n)}\laplace K\mathop{\Big[}1-e^{\sigma t}\mathop{\Big[}cos(\omega t)-\frac{\sigma}{\omega}sin(\omega t)\mathop{\Big]}\mathop{\Big]}=y(t)}
%\end{equation}
%\begin{multicols}{2}
%\begin{itemize}
%\item K ergibt sich durch $y_\infty$.
%\item $\omega$ ergibt sich durch die Periode $T_\omega$ = 2$\pi$/$\omega$ [= $2T_m$].
%\item $\sigma$ ergibt sich durch die Überschwingweite ym aus der Formel $y_m/y_\infty = e^{
%\frac{\sigma}{\omega}\pi}$.
%\item $\omega$n und $\xi$ ergeben sich gemäss (\ref{xiOmega}) aus $\omega$ und $\sigma$.
%\end{itemize}
%	\columnbreak
%\begin{center}
%	\includegraphics[width=7cm]{./images/pt2.png}
%\end{center}
%\end{multicols}

\subsubsection{Totzeit, Padé-Approximation \buchSeite{45}}
\begin{equation}
\boxed{e^{-sT_t}=e^x\mathop\mid\limits{_{x=-sT_t}} \quad mit \quad e^x\approx 1 \approx \frac{2+x}{2-x} \approx \frac{12+6x+x^2}{12-6x+x^2} \approx \ldots}
\end{equation}
\begin{itemize}
\item  Sie ist stabil, unabhängig von der Ordnung.
\item  Die Nullstellen der Approximation entsprechen — an der Imaginärachse der
komplexen Ebene gespiegelt — den Polen. Beim Ersetzen eines Totzeitglieds
bleibt damit die Allpasseigenschaft erhalten mit $|G(j\omega)| = 1$. Weiter bleibt
auch die Nichtminimalphasigkeit erhalten.
\item  Ihr Phasengang konvergiert mit zunehmender Ordnung gegen denjenigen des
Totzeitglieds $e^{-sT_t} : \angle e^{-j\omega T_t} = -\omega T_t$.
\end{itemize}

\subsection{Blockdiagramm Algebra}
\begin{center}
	\includegraphics[width=16cm]{./images/blockdiagrammAlgebra.png}
\end{center}
\vspace{-1cm}\begin{multicols}{2}
\begin{center}
\includegraphics[width=9cm]{./images/blockdiagrammAlgebra2.png}
\end{center}

\columnbreak

\begin{itemize}
\item Zwei Blöcke hintereinander 
\begin{itemize}
	\item $G(s)=G_1(s)\cdot G_2(s)$
	\item $Y(s) = G(s)=G_1(s)\cdot G_2(s) \cdot U(s)$
\end{itemize}
\item Zwei Blöcke Parallel
\begin{itemize}
	\item $G(s) = G_1(s) + G_2(s)$
	\item $Y(s)= (G_s(s)+G_2(s))\cdot U(s)$
\end{itemize}

\item Kreisschaltung
\begin{itemize}
	\item $G(s) = \frac{G_1(s)}{1\pm G_1(s)\cdot G_2(s)}$
	\item $Y(s) = \frac{G_1(s)}{1\pm G_1(s)\cdot G_2(s)} \cdot U(s)$
\end{itemize}

\item Kreisschaltung ohne $G_2(s)$ in Rückkopplung
\begin{itemize}
	\item $G(s) = \frac{Z_1(s)}{N_1(s)\pm Z_1(s)}$
\end{itemize}

\end{itemize}
\end{multicols}

\subsection{Analog Rechner \buchSeite{57}}
	\begin{center}
	\includegraphics[width=12cm]{./images/AnalogRechner}
	\end{center}

\subsection{Bodediagramm und Nyquistdiagramm \buchSeite{63}}
\subsubsection{Einleitung}
\begin{itemize}
	\item Ist der offene Regelkreis stabil, muss man aufpassen, dass durch das Schliessen
	des Regelkreises (bzw. der Kreisschaltung) der geschlossene Regelkreis nicht instabil
	wird.
	\item Ist dagegen der offene Regelkreis instabil, dann kann untersucht werden, unter
	welchen Bedingungen der geschlossenen Regelkreis stabil werden kann.
	\item Das
	spezielle (oder vereinfachte) Nyquistkriterium kann nur dann angewendet werden,
	wenn der offene Regelkreis ein System mit Ausgleich ist;
	\item Beim allgemeinen Nyquistkriterium
	kann der offene Regelkreis ein beliebiges LZI-System sein, insbesondere also auch
	ein instabiles.
\end{itemize}
\subsubsection{Vereinfachte Nyquistkriterium}
		Der geschlossene Regelkreis ist genau dann stabil, wenn beim Durchlauf der
		Ortskurve von $G_0$ in Richtung zunehmender Frequenz der kritische Punkt -1 \glqq zur
		Linken\grqq\ liegt.

\subsubsection{Anwendung des Allgemeinen Nyquistkriteriums \buchSeite{73}}
\begin{enumerate}
\item Die Anzahl N der Pole von $G_0$, die in der rechten Halbebene oder auf der
imaginären Achse liegen, muss bestimmt werden. $(0 \leq N \leq n)$
\item Beide Äste ($-\infty < \omega < +\infty$) der Ortskurve von $G_0(j\omega)$ müssen gezeichnet werden.
\item Umschlingt die Ortskurve den kritischen Punkt -1 genau N mal positiv (im
Gegenuhrzeigersinn), dann ist der geschlossene Regelkreis 
$\frac{Y(s)}{R(s)} = \frac{G_0}{1+G_0}$ asymptotisch stabil.
\end{enumerate}
%Beispiel: $G_0(S)= K\cdot\frac{1}{s}\frac{1}{(s+1)^2}$
%Wegen des Integrators ist N = 1. Dies bedeutet, dass der kritische Punkt
%einmal im Gegenuhrzeigersinn umschlungen werden muss.\\
%\begin{tabularx}{\textwidth}{XXX}
%$K > 2$ & Anzahl Umschlingungen: -1 & Regelkreis instabil \\
%$2 > K > 0$ & Anzahl Umschlingungen: 1 & Regelkreis (asymptotisch) stabil\\
%$0 >K$ & Anzahl Umschlingungen: 0 & rightarrow Regelkreis instabil\\
%\end{tabularx}

\subsection{Modellreduktion \buchSeite{76}}

Tendenziell versucht man bei einer Modellreduktion, die tieffrequenten Eigenschaften
beizubehalten und dafür Ungenauigkeiten in den höheren Frequenzbereichen in
Kauf zu nehmen. Beim Modell $G_M$ wird also am ehesten $T_2$ als vernachlässigbar
klein betrachtet, womit sich folgendes reduzierte Modell $\overline{G}_M$ ergibt
\[G_M(s)=K\cdot\frac{1}{1+T_1s}\cdot\frac{1}{1+2\xi T_2 s +T^2_2 s^2} \quad \text{wird zu} \quad \overline{G}_M(s)=K\cdot\frac{1}{1+T_1s}\]